\documentclass[12 pt,fullpage]{article}
%\documentclass[draft]{ectaart}
%%%%%%%%%%%%%%%%%%%%%%%%%%%%%%%%%%%%%%%%%%%%%%%%%%%%%%%%%%%%%%%%%%%%%%%%%%%%%%%%%%%%%%%%%%%%%%%%%%%%%%%%%%%%%%%%%%%%%%%%%%%%%%%%%%%%%%%%%%%%%%%%%%%%%%%%%%%%%%%%%%%%%%%%%%%%%%%%%%%%%%%%%%%%%%%%%%%%%%%%%%%%%%%%%%%%%%%%%%%%%%%%%%%%%%%%%%%%%%%%%%%%%%%%%%%%
\usepackage{graphicx}
\usepackage{amsfonts}
\usepackage{amssymb}
\usepackage{amsmath}
\usepackage{amsthm}
\usepackage{textcomp}
\usepackage{ifsym}
\usepackage{mathrsfs}
\usepackage{epstopdf}
\usepackage{MnSymbol}
\usepackage{multirow}
\usepackage{epsfig}
\usepackage[normalem]{ulem}
\usepackage[round]{natbib}
\usepackage{array}
\usepackage{caption}
\usepackage{subcaption}
\usepackage{url}
\usepackage{placeins}
\usepackage{longtable}
\usepackage{pdflscape}
\usepackage{tabularx}
\usepackage{booktabs}
%\usepackage{xcolor}
\usepackage{bbm}
\usepackage[colorlinks,citecolor=blue]{hyperref}
\usepackage[margin=1.1in]{geometry}
\usepackage{hyperref}
\usepackage{comment}
\usepackage{chngcntr}
\usepackage{amsmath}
\usepackage{setspace}
%\usepackage{colortbl}
\usepackage[table,xcdraw]{xcolor}

%\usepackage[bottom]{footmisc}
\usepackage[hang,flushmargin]{footmisc}
\counterwithout{equation}{section} 
%\counterwithin{equation}{chapter} 
%\numberwithin{equation}{section}

\newcommand{\footremember}[2]{%
	\footnote{#2}
	\newcounter{#1}
	\setcounter{#1}{\value{footnote}}%
}
\newcommand{\footrecall}[1]{%
	\footnotemark[\value{#1}]%
}

\setcounter{MaxMatrixCols}{10}

\linespread{1.5}



%\topmargin=-1.8cm \textheight=23.8cm \oddsidemargin=-0.3cm
%\evensidemargin=-0.5cm \textwidth=17.1cm

\DeclareMathOperator*{\argmax}{arg\,max}
\DeclareMathOperator*{\argmin}{arg\,min}
\theoremstyle{plain}
\newtheorem{theo}{Theorem}
\newtheorem{conj}{Conjecture}
\newtheorem{prop}{Proposition}
\newtheorem*{prop*}{Proposition}
\newtheorem{coro}{Corollary}
\newtheorem{lemma}{Lemma}
\newtheorem*{lemma*}{Lemma}
\newtheorem{assum}{Assumption}
\newtheorem{defi}{Definition}

\newtheorem{innercustomgeneric}{\customgenericname}
\providecommand{\customgenericname}{}
\newcommand{\newcustomtheorem}[2]{%
	\newenvironment{#1}[1]
	{%
		\renewcommand\customgenericname{#2}%
		\renewcommand\theinnercustomgeneric{##1}%
		\innercustomgeneric
	}
	{\endinnercustomgeneric}
}

\newcustomtheorem{customprop}{Proposition}
\newcustomtheorem{customlemma}{Lemma}

\def\sym#1{\ifmmode^{#1}\else\(^{#1}\)\fi}


\newcommand{\figtext}[1]{
	\vspace{-1.9ex}
	\captionsetup{justification=justified,font=footnotesize}
	\caption*{\hspace{6pt}\hangindent=1.5em #1}
}
\newcommand{\fignote}[1]{\figtext{\emph{Note:~}~#1}}

\newcommand{\figsource}[1]{\figtext{\emph{Source:~}~#1}}

\newcommand{\starnote}{\figtext{* p $<$ 0.1, ** p $<$ 0.05, *** p $<$ 0.01. Standard errors in parentheses.}}

\title{\textbf{Fentanyl Consumption and Peasants Income}}
\author{\vspace{-30mm}}

\begin{document}
	\date{\vspace{-25mm}}
	\maketitle
	\vspace{-15mm}



\newpage
\subsection*{Preliminar regression}

\begin{table}[h!]
	\begin{center}
		\scalebox{0.65}{
			\begin{tabular}{lccccc} \\ \hline 
                    &\multicolumn{1}{c}{(1)}         &\multicolumn{1}{c}{(2)}         &\multicolumn{1}{c}{(3)}         &\multicolumn{1}{c}{(4)}         &\multicolumn{1}{c}{(5)}         \\
 & \multicolumn{5}{c}{Income} \\ \cline{2-6} & Survey & Avg &  \multicolumn{1}{c}{Small} & \multicolumn{1}{c}{Medium}& \multicolumn{1}{c}{Big} \\ \hline   &  &  &  &  &  \\ \textbf{Panel A: Full sample} \\ & & & & & \\
Log number of deaths (US) caused, among others, by fentanyl&      -0.009         &       0.032         &       0.014         &       0.015         &      -0.564         \\
                    &     (0.048)         &     (0.021)         &     (0.016)         &     (0.047)         &     (0.383)         \\
\addlinespace
Log number of deaths (US) caused, among others, by fentanyl&       0.007         &      -0.107         &      -0.159\sym{***}&      -0.393\sym{***}&       1.240         \\
                    &     (0.084)         &     (0.067)         &     (0.034)         &     (0.084)         &     (1.019)         \\
\addlinespace
State CPI           &       0.008         &      -0.001         &      -0.006\sym{***}&      -0.016\sym{***}&      -0.122         \\
                    &     (0.006)         &     (0.004)         &     (0.001)         &     (0.003)         &     (0.067)         \\
\addlinespace
Participation rate  &       0.000         &      -0.002         &       0.000         &       0.000         &      -0.001         \\
                    &     (0.002)         &     (0.003)         &     (0.000)         &     (0.001)         &     (0.015)         \\
\addlinespace
Log Exchange rate (COP/USD)&      -0.055         &       0.225         &       0.197\sym{***}&       0.584\sym{***}&       2.745\sym{**} \\
                    &     (0.091)         &     (0.157)         &     (0.014)         &     (0.050)         &     (1.039)         \\
\addlinespace
Economic Performance Index&       0.008\sym{***}&      -0.003\sym{**} &       0.004\sym{***}&       0.010\sym{***}&       0.031         \\
                    &     (0.002)         &     (0.001)         &     (0.000)         &     (0.001)         &     (0.021)         \\
\arrayrulecolor{black!10}\midrule
Observations        &         950         &         819         &         819         &         819         &         819         \\
R-squared           &       0.556         &       0.342         &       0.518         &       0.576         &       0.837         \\
State F.E.          &$\checkmark$         &$\checkmark$         &$\checkmark$         &$\checkmark$         &$\checkmark$         \\
Month F.E.          &$\checkmark$         &$\checkmark$         &$\checkmark$         &$\checkmark$         &$\checkmark$         \\


 \hline  &  &  &  &  &  \\ \textbf{Panel B: Sub sample from 2019} \\ & & & & & \\
Log number of deaths (US) caused, among others, by fentanyl&      -0.084         &      -0.202\sym{**} &      -0.253\sym{***}&      -0.592\sym{**} &      -1.253         \\
                    &     (0.078)         &     (0.079)         &     (0.062)         &     (0.174)         &     (1.544)         \\
\addlinespace
Log number of deaths (US) caused, among others, by fentanyl&       0.172         &       0.113         &       0.277\sym{**} &       0.661         &       2.834         \\
                    &     (0.229)         &     (0.089)         &     (0.113)         &     (0.353)         &     (2.522)         \\
\addlinespace
State CPI           &       0.006         &       0.009\sym{*}  &      -0.003         &      -0.010         &      -0.110         \\
                    &     (0.020)         &     (0.004)         &     (0.005)         &     (0.013)         &     (0.168)         \\
\addlinespace
Participation rate  &       0.003         &      -0.000         &      -0.001         &      -0.002         &      -0.010         \\
                    &     (0.003)         &     (0.001)         &     (0.000)         &     (0.001)         &     (0.011)         \\
\addlinespace
Log Exchange rate (COP/USD)&       0.141         &       0.482\sym{**} &      -0.284         &      -0.672         &      -0.172         \\
                    &     (0.411)         &     (0.155)         &     (0.202)         &     (0.433)         &     (4.611)         \\
\addlinespace
Economic Performance Index&       0.007\sym{**} &      -0.001         &      -0.001         &      -0.003         &       0.017         \\
                    &     (0.003)         &     (0.001)         &     (0.002)         &     (0.004)         &     (0.036)         \\
\arrayrulecolor{black!10}\midrule
Observations        &         362         &         252         &         252         &         252         &         252         \\
R-squared           &       0.525         &       0.765         &       0.672         &       0.671         &       0.871         \\
State F.E.          &$\checkmark$         &$\checkmark$         &$\checkmark$         &$\checkmark$         &$\checkmark$         \\
Month F.E.          &$\checkmark$         &$\checkmark$         &$\checkmark$         &$\checkmark$         &$\checkmark$         \\
\arrayrulecolor{black}\bottomrule
\multicolumn{6}{c}{*** p$<$0.01, ** p$<$0.05, * p$<$0.1}
\end{tabular}
}
		\caption{}
	\end{center}
\end{table}
\vspace{-3mm}
\footnotesize Note: This table reports the regression of income as a function of fentanyl and cocaine related deaths in the US. Column 1 shows the dependent variable as the traditional approach using survey data for average rural household income. Columns 2-5 use satellite light intensity data, in particular column 2 uses the average light intensity, column 3 establishes a classification of small villages. column 4 for medium-size villages and column 5 for big cities. All columns include state and monthly fixed effects and control by standard macroeconomic variables that affects the business cycle. Unit of observation is state x month. 


\end{document}